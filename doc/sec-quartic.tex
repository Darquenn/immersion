\section{Quartic: degree=4}
\label{sec.quartic}

\subsection{Objectives}
The assignment in this section is:
\begin{enumerate}
\item Complete the file \code{src/quartic.py} by writing a Python
  function which computes and returns the x-intercepts of a curve
  equation is \[y=a x^4 + b x^3 + c x^2 + d x + e\,.\]
\item You'll need to replace all occurances of \code{raise NotImplementedError()}
  with correct python code which passes the tests.
\item Test the function in GitHub Codespaces by running the file\\
  \code{test/test\_quartic.py}.
\end{enumerate}

\subsection{Overview}

We wish to find the roots of the quartic polynomial given by 
\[P(x) = ax^4 + b x^3 + c x^2 + d x + e\,.\]

We will do this by first finding
(approximating) a root, $r$, using a technique called \emph{binary
seach}.  We will determine which regions to search by examining the derivative

\[\frac{d}{dx} P(x) = 4ax^3 + 3b x^2 + 2c x + d\,.\]

Knowing root $r$, we know that $(x-r)$ is a factor of 
$a x^4 + b x^3 + c x^2 + d x + e$.  Once $(x-r)$ is factored out of $P(x)$
what remains is a cubic polynomial of the form $A x^3 + B x^2 + C x + D$.
We can find the roots of the cubic using the technique
explained in Section~\ref{sec.cubic}, thus obtaining the roots of
the quartic polynomial.


\subsection{The Math}


A polynomial of degree 4 has the form $P(x) = a x^4 + b x^3 + c x^2 + d x + e$. If $a=0$ than $P(x)$ is really
a cubic polynomial (degree 3) and can be solved using the techniques described in Section~\ref{sec.cubic}.

If $a<0$ then similar to what we did for the cubic polynomial in Section~\ref{sec.cubic.math}, we instead find the
roots of $-P(x) = -a x^4 - b x^3 - c x^2 - d x - e$.  $P(x)$ and $-P(x)$ are guaranteed to have the same roots,
but the code in Python will be easier if we know that $a >= 0$.


Two polynomial equations (for $P_1(x)$ and $\frac{d}{dx} P_1(x)$) are plotted in Figure~\ref{fig.quartic}.

\begin{figure}
\centering
%% derived from https://tex.stackexchange.com/questions/357538/graph-of-a-parabola-on-pgfplots
%% Thanks to Stefan Pinnow
%%     https://tex.stackexchange.com/users/95441/stefan-pinnow

\begin{figure}
\centering
\begin{tikzpicture}
    \begin{axis}[
        width=3in,
        height=3in,
        axis lines=middle,
        xmin=-5,
        xmax=12,
        ymin=-60,
        ymax=60,
%        xtick={20000},
        % ---------------------------------------------------------------------
        % you don't want the ticks/tick labels to be scaled
        scaled ticks=false,
%        % and the tick labels are shown by default the way you want them,
%        % so you don't need to specify them explicitely
%        xticklabels={$20000$},
        % ---------------------------------------------------------------------
 %       ytick={\empty},
        ticklabel style={font=\scriptsize},
        xlabel=$x$,
        ylabel=$y$,
        axis line style={
            latex-latex,
            shorten >=-12.5pt,
            shorten <=-12.5pt,
        },
        xlabel style={at={(ticklabel* cs:1)}, xshift=12.5pt, anchor=north west},
        ylabel style={at={(ticklabel* cs:1)}, yshift=12.5pt, anchor=south west},
    ]

        \addplot[samples=51,smooth,domain=-4:12,color=blue] {0.125 * x^4 -2* x^3 + 7 * x^2 + x + 6};

    \end{axis}


\end{tikzpicture}
%
\begin{tikzpicture}
    \begin{axis}[
        width=3in,
        height=3in,
        axis lines=middle,
        xmin=-5,
        xmax=12,
        ymin=-60,
        ymax=60,
%        xtick={20000},
        % ---------------------------------------------------------------------
        % you don't want the ticks/tick labels to be scaled
        scaled ticks=false,
%        % and the tick labels are shown by default the way you want them,
%        % so you don't need to specify them explicitely
%        xticklabels={$20000$},
        % ---------------------------------------------------------------------
 %       ytick={\empty},
        ticklabel style={font=\scriptsize},
        xlabel=$x$,
        ylabel=$y$,
        axis line style={
            latex-latex,
            shorten >=-12.5pt,
            shorten <=-12.5pt,
        },
        xlabel style={at={(ticklabel* cs:1)}, xshift=12.5pt, anchor=north west},
        ylabel style={at={(ticklabel* cs:1)}, yshift=12.5pt, anchor=south west},
    ]

        \addplot[samples=51,smooth,domain=-4:12,color=blue] {-0.125 * x^4 +2* x^3 - 7 * x^2 - x - 6};

    \end{axis}


\end{tikzpicture}
\caption{Quartics: [left] ${y=P_1(x) = \frac{1}{8}  x^4 -2 x^3 + 7  x^2 + x + 6}$ \\[2pt]
[right] ${y=P_2(x) = -P_1(x) = -\frac{1}{8}  x^4 +2 x^3 - 7  x^2 - x - 6}$}
\label{fig.quartic}
\end{figure}

\caption{Quartics}
\label{fig.quartic}
\end{figure}

\begin{align*}
  P_1(x) &= \frac{1}{8}  x^4 -2 x^3 + 7  x^2 + x + 6\\
  \frac{d}{dx} P_1(x) &= \frac{1}{2} * x^3 -6* x^2 + 14 * x + 1
\end{align*}

Notice that at the extreme left and right, the plot of $P_1(x)$ opens
upward, and the graph changes direction 3 times.  The values of $x$ at
which $P_1(x)$ changes direction are called the \emph{inflection
points}.  The inflection points are local minima of $P_1(x)$, and can
be computed by finding the values for which $\frac{d}{dx} P_1(x) = 0$.
Since the derivative of a degree 4 polynomial is a degree 3 (cubic)
polynomial, we can find where $\frac{d}{dx} P_1(x) = 0$ by using the
techniques explained in Section~\ref{sec.cubic}.

Once we have determined the 3 inflection points $x_1$, $x_2$, and $x_3$, we can
