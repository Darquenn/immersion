%% The definitions in this file are inspired from stackexchange post
%% https://tex.stackexchange.com/questions/668903/how-to-prevent-thmbox-from-rendering-over-page-number/668933#668933
%% Thanks to https://tex.stackexchange.com/users/264024/udi-fogiel
%% Udi Fogiel


\usepackage[most]{tcolorbox}

\newlength{\thmboxvlineindent}
\setlength{\thmboxvlineindent}{\dimexpr21.14975pt-0.4em-0.3pt}


\definecolor{almond}{rgb}{0.94, 0.87, 0.8}
\definecolor{beige}{rgb}{0.96, 0.96, 0.86}
\definecolor{blond}{rgb}{0.98, 0.94, 0.75}

\definecolor{dfncolor}{rgb}{1.0, 0.92, 0.85}
\definecolor{excolor}{rgb}{0.92, 1.0, 0.85}
\definecolor{thcolor}{rgb}{0.85, 0.92, 1.0}

\definecolor{mint}{rgb}{0.77, 0.94, 0.86}
\definecolor{propositioncolor}{rgb}{1.0, 0.8, 0.8}%255,204,204
\tcbset{
    thmbox/.style={
        enhanced,
        breakable,
        sharp corners=all,
        fonttitle=\bfseries,
        fontupper=\normalfont,
        top=0mm,
        bottom=0mm,
        right=0mm,
        left=\dimexpr\thmboxvlineindent-0.3pt,
        boxsep=0.4em,
        colback=white,
        colframe=black,
        colbacktitle=white,
        coltitle=black,
        attach boxed title to top left,
        boxed title style={empty, size=minimal, bottom=2.5pt},
        before upper={\parindent=21.14975pt\noindent},
        left skip=\parskip,
        %% overlay unbroken ={
        %%     \draw[line width=0.6pt] (title.south west)--(title.south east);
        %%     \draw[line width=0.6pt] ([xshift=\thmboxvlineindent]frame.north west)--([xshift=\thmboxvlineindent]frame.south west)--++(0:0.3pt);
        %%     \draw[line width=0.6pt] ([xshift=\thmboxvlineindent]frame.south west)--++(0:1cm);},
        %% overlay first={
        %%     \draw[line width=0.6pt] (title.south west)--(title.south east); 
        %%     \draw[line width=0.6pt] ([xshift=\thmboxvlineindent]frame.north west)--([xshift=\thmboxvlineindent]frame.south west);},
        %% overlay middle={
        %%     \draw[line width=0.6pt] ([xshift=\thmboxvlineindent]frame.north west)--([xshift=\thmboxvlineindent]frame.south west);},
        %% overlay last={
        %%     \draw[line width=0.6pt] ([xshift=\thmboxvlineindent]frame.north west)--([xshift=\thmboxvlineindent]frame.south west)--++(0:0.3pt);
        %%     \draw[line width=0.6pt] ([xshift=\thmboxvlineindent]frame.south west)--++(0:1cm);}
    }
}

\newtcbtheorem[number within=section]
{theorem}
{Theorem}
{thmbox,description font=\itshape,colback=thcolor,description delimiters={{\normalfont\bfseries(}}{\normalfont\bfseries)},separator sign none,label separator=.}{th} % the last argument here, i.e. "theo", is the prefix for the label.

\newtcbtheorem[use counter from=theorem]
{definition}
{Definition}
{thmbox,description font=\itshape,colback=dfncolor,description delimiters={{\normalfont\bfseries(}}{\normalfont\bfseries)},separator sign none,label separator=.}{def}

\newtcbtheorem[use counter from=theorem]
{example}
{Example}
{thmbox,description font=\itshape,colback=excolor,description delimiters={{\normalfont\bfseries(}}{\normalfont\bfseries)},separator sign none,label separator=.}{ex}

\newtcbtheorem[use counter from=theorem]
{corollary}
{Corollary}
{thmbox,description font=\itshape,colback=blond,description delimiters={{\normalfont\bfseries(}}{\normalfont\bfseries)},separator sign none,label separator=.}{cor}

\newtcbtheorem[use counter from=theorem]
{lemma}
{Lemma}
{thmbox,description font=\itshape,colback=mint,description delimiters={{\normalfont\bfseries(}}{\normalfont\bfseries)},separator sign none,label separator=.}{lem}

%% TODO choose color uniquely
\newtcbtheorem[use counter from=theorem]
{proposition}
{Proposition}
{thmbox,description font=\itshape,colback=propositioncolor,description delimiters={{\normalfont\bfseries(}}{\normalfont\bfseries)},separator sign none,label separator=.}{prop}

\newtcbtheorem[use counter from=theorem]
{listing}
{Listing}
{thmbox,description font=\itshape,description delimiters={{\normalfont\bfseries(}}{\normalfont\bfseries)},separator sign none,label separator=.}{list}


\definecolor{codegreen}{rgb}{0,0.6,0}
\definecolor{codegray}{rgb}{0.5,0.5,0.5}
\definecolor{codepurple}{rgb}{0.58,0,0.82}
\definecolor{codebackcolor}{rgb}{0.95,0.92,0.92}
\lstdefinestyle{mypython}{
    language=Python,
    backgroundcolor=\color{codebackcolor},   
    commentstyle=\color{codegreen},
    keywordstyle=\color{magenta},
    morekeywords={nonlocal},
    numberstyle=\tiny\color{codegray},
    stringstyle=\color{codepurple},
    basicstyle=\ttfamily\footnotesize,
    breakatwhitespace=false,         
    breaklines=true,                 
    captionpos=b,                    
    keepspaces=true,                 
    numbers=left,                    
    numbersep=5pt,                  
    showspaces=false,                
    showstringspaces=false,
    showtabs=false,                  
    tabsize=2
}
\lstset{style=mypython}

