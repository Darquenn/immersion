\section{The Line: degree=1}
\label{sec.line}

%% derived from https://tex.stackexchange.com/questions/357538/graph-of-a-parabola-on-pgfplots
%% Thanks to Stefan Pinnow
%%     https://tex.stackexchange.com/users/95441/stefan-pinnow

\begin{figure}
\centering
\begin{tikzpicture}
    \begin{axis}[
        width=3in,
        height=3in,
        axis lines=middle,
        xmin=-10,
        xmax=10,
        ymin=-10,
        ymax=10,
%        xtick={1,1},
        % ---------------------------------------------------------------------
        % you don't want the ticks/tick labels to be scaled
        scaled ticks=false,
%        % and the tick labels are shown by default the way you want them,
%        % so you don't need to specify them explicitely
%        xticklabels={$20000$},
        % ---------------------------------------------------------------------
%        ytick={\empty},
        ticklabel style={font=\scriptsize},
        xlabel=$x$,
        ylabel=$y$,
        axis line style={
            latex-latex,
            shorten >=-12.5pt,
            shorten <=-12.5pt,
        },
        xlabel style={at={(ticklabel* cs:1)}, xshift=12.5pt, anchor=north west},
        ylabel style={at={(ticklabel* cs:1)}, yshift=12.5pt, anchor=south west},
    ]

        \addplot[samples=51,smooth,domain=-10:10] {-1.2*x + 3};
    \end{axis}

\end{tikzpicture}

\caption{Line: $y = P(x) = -1.2*x + 3$}
\label{fig.line}
\end{figure}


A polynomial of degree 1 is a function $P(x)=a x + b$ whose graph is a line.   If $a=0$ the
line is horizontal an crosses the y-axis at $y=b$; thus if $b\neq 0$ then $P(x)\neq 0$.
So if $a=0$ we will assume that $P(x)$ has no root.   

\subsection{Mathematical computation of x-intercept of line}


Figure~\ref{fig.line} shows the graphs of two polynomials of degree 1.

\begin{align*}
  P_1(x) &= -1.2 x + 3\\
  P_2(x) &= 0 x + 6
\end{align*}

The line representing $P_1(x)$ has an x-intercept computed as in Example~\ref{ex.line}.


\begin{example}{Computing x-intercept of a line}{line}
  \begin{align*}
  P_1(x) &= a x + b \\
  0  &= -1.2 x + 3\\
  x &= \frac{3}{1.2} = 2.5
  \end{align*}
\end{example}

We see in Figure~\ref{fig.line} that if the coefficient of $x$ is 0, then the
line is horizontal and has no x-intercept.  If the horizontal line is coincident
with the x-axis, \ie, if $a=0$ and $b=0$, then there is no unique x-intercept.





Otherwise we have a line as shown in Figure~\ref{fig.line}.  We can
solve for the x-intercept by setting y to 0 and solving for x.
\begin{align*}
  y &= a x + b\\
  0 &= a x + b\\
  -b &= a x\\
  \frac{-b}{a} &= x
\end{align*}


You should update the code in the file \code{src/line.py} so that if
$a$ is 0, then the function returns an empty list and otherwise
returns a list of the value~$\frac{-b}{a}$.  You must figure out how
to write that in the Python language.  \Ie, you must figure out how to
divde two numbers in Python and how to create a list, empty and
otherwise.

\subsection{Programmatic computation of x-intercept of line}

You will develop the code in \code{src/line.py}.

\begin{listing}{Function declaration to find x-intercept of a line.}{code.line}
\begin{minipage}[c]{0.95\textwidth}\begin{lstlisting}
def find_x_intercept(a,b):
    if a == 0:
        ... 
    else:
        ...
\end{lstlisting}\end{minipage}\end{listing}


