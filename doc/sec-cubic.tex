\section{Cubic: degree=3}
\label{sec.cubic}

A polynomial of degree 3 has the form $P(x) = a x^3 + b x^2 + c x + d$. If $a=0$ than $P(x)$ is really
a quadratic polynomial (degree 2) and can be solved using the techniques described in Section~\ref{sec.quadratic}.  Every polynomial of degree 3
with real coefficients (for which $a\neq 0$) has at least one real root because 
\begin{enumerate}
\item Either $\lim\limits_{x\to-\infty}P(x) = -\infty$ and $\lim\limits_{x\to\infty}P(x) = \infty$,
  \item Or $\lim\limits_{x\to-\infty}P(x) = \infty$ and $\lim\limits_{x\to\infty}P(x) = -\infty$.
\end{enumerate}


We have a two cubic equations plotted in Figure~\ref{fig.cubic}.
%% derived from https://tex.stackexchange.com/questions/357538/graph-of-a-parabola-on-pgfplots
%% Thanks to Stefan Pinnow
%%     https://tex.stackexchange.com/users/95441/stefan-pinnow

\begin{figure}
\centering
\begin{tikzpicture}
    \begin{axis}[
        width=3in,
        height=3in,
        axis lines=middle,
        xmin=-5,
        xmax=6,
        ymin=-50,
        ymax=60,
%        xtick={20000},
        % ---------------------------------------------------------------------
        % you don't want the ticks/tick labels to be scaled
        scaled ticks=false,
%        % and the tick labels are shown by default the way you want them,
%        % so you don't need to specify them explicitely
%        xticklabels={$20000$},
        % ---------------------------------------------------------------------
 %       ytick={\empty},
        ticklabel style={font=\scriptsize},
        xlabel=$x$,
        ylabel=$y$,
        axis line style={
            latex-latex,
            shorten >=-12.5pt,
            shorten <=-12.5pt,
        },
        xlabel style={at={(ticklabel* cs:1)}, xshift=12.5pt, anchor=north west},
        ylabel style={at={(ticklabel* cs:1)}, yshift=12.5pt, anchor=south west},
    ]

        \addplot[samples=51,smooth,domain=-10:10] {x^3 - 4*x^2 - 2*x - 10};

    \end{axis}


\end{tikzpicture}
\caption{Cubic: $y=P(x) = x^3 - 4*x^2 - 2*x - 10$}
\label{fig.cubic}
\end{figure}

\begin{align*}
  P_1(x) &= x^3 - 4*x^2 - 2*x - 10\\
  P_2(x) = -P_1(x) &= -x^3 + 4*x^2 + 2*x + 10
\end{align*}

$P_1(x)$ exemplifies the \emph{standard} case where the leading coefficient is positive: $a>0$.
$P_2(x)$ exemplifies the alternate case where the leading coefficient is negative: $a<0$.
However, we notice that $P_1(x)$ and $-P_1(x)$ have the exact same roots, because if $P_2(x) = -P_1(x) = 0$,
then $P_1(x) = 0$.  Therefore, if $a<0$, we can simply find the roots of $-a x^3 -b x^2 - c x - d$;
\ie, we simply negate the coefficients and find the roots of the negated cubic polynomial.


\subsection{Programmatically Computing Roots of a Quadratic}

Steps for computing roots of a cubic polynomial.

\begin{enumerate}
\item Assume the coefficients of $P(x) = a x^3 + b x^2 + c x + d$ are
  \code{a},  \code{b},  \code{c},  and~\code{d}.
\item If \code{a==0}, then delegate to the previous solution by
  calling \code{find\_quadratic\_roots} and returning its return
  value.  Be careful, \code{find\_quadratic\_roots} accepts 3 input
  parameters.
\item If $P(x)$ is of the form of $P_2(x)$ in Figure~\ref{fig.cubic},
  \ie, if $a<0$, then compute and return the roots of $-P(x)$.  To
  compute the roots of $-P(x)$ we simply return
  \code{find\_cubic\_roots(-a, -b, -c, -d)}.
\item Since $P(0) = d$, then we can easily evaluate the polynomial at 0 to get its y-intercept.
  \begin{enumerate}
  \item If $d = 0$, then 0 is a root, $P(0) = 0$.
  \item If $d>0$ then there is a root on the negative x-axis.  Find it with a binary search.
  \item If $d<0$ then there is a root on the positive x-axis. Find it with a binary search.
  \end{enumerate}
\item See Section~\ref{sec.binary.search} to explain the binary search.
\item Once the root $r$ is found, this means $(x-r)$ factors out $P(x)$. \Ie,
  \begin{align*}
    P(x) &= (x-r)(A x^2 + B x + C)\\ 
    A &= a\\
    B &= b + a r \\
    &= b + A r\\
    C &= c + b r + a r^2 \\
    &=
    c + B r\\
  \end{align*}
\item The roots of the cubic are \code{[r]} concatenated to the roots of the quadratic 
  $A x^2 + B x + C$, which you can compute using the techniques in Section~\ref{sec.quadratic}.
\end{enumerate}

\begin{listing}{Function to compute roots of cubic polynomial.}{code.cubic}
\begin{minipage}[c]{0.98\textwidth}\begin{lstlisting}
def find_cubic_roots(a, b, c, d):
    epsilon = 0.00001
    if a == 0:
        return find_quadratic_roots(b, c, d)
    if a < 0:
        return find_cubic_roots(-a, -b, -c, -d)

    f = make_cubic(a, b, c, d)

    # f(0) = d
    if d == 0.0:
        r = 0.0
    elif d > 0: # f(0) > d, so f(-infinity) < 0 and f(infinity) > 0
        r = search_root_left(-1, 0, f, epsilon)
    else:
        r = search_root_right(0, 1, f, epsilon)
    return factor_out_cubic_root(r, a, b, c)
\end{lstlisting}\end{minipage}\end{listing}

\subsection{Find a root using Binary Search}
\label{sec.binary.search}
