\section{Introduction}
\label{sec.intro}

In this atelier you will experiment with software development
in the cloud using GitHub and the Python programming language.

\subsection{Overview}

You will finish the development of a program which will compute the
roots of polynomials of degree 1 through 5.
\begin{align*}
  a x + b\\
  a x^2 + b x + c\\
  a x^3 + b x^2 + c x + d\\
  a x^4 + b x^3 + c x^2 + d x + e\\
  a x^5 + b x^4 + c x^3 + d x^2 + e x + f
\end{align*}

The suite of Python functions you will implement (finish the
implementation of) are designed to review your mastery of
\begin{itemize}
\item Polynomials
\item Factorization
\item Quadratic formula
\item Finding Extrema using the Derivative
\item and more
\end{itemize}

\subsection{Flow of Atelier}
\begin{enumerate}
\item \textbf{ Set up your cloud environment:}

  In Section~\ref{sec.github} you will setup your cloud environment. You
  will then examine and modify a simply \code{hello world} program.

\item \textbf{Optional: Review of Algebra}

  Section~\ref{sec.general} which explains the math
  connected to the program you will develop.

\item \textbf{Hello World:}

  Starting in Section~\ref{sec.hello.world}, write the
  \emph{Hello World} program to help you learn how to use the GitHub
  Code Space programming environment.  Complete file \code{hello.py}.

In the remaining sections, starting with Section~\ref{sec.line}, you
will develop Python code to compute the roots of polynomials.  

\item \textbf{Roots of a line:}

  In Section~\ref{sec.line} you'll find the sole root of a 1st
  degree polynomial (called a line).  Complete file \code{line.py}.
  Use Section~\ref{sec.line} to understand the theory.

\item \textbf{Roots of a quadratic:}

  You'll complete the file
  \code{quadratic.py}.  You'll find at most two roots of a 2nd degree
  polynomial (called a quadratic).  Use Section~\ref{sec.quadratic} to
  understand the theory.

\item \textbf{Roots of a cubic:}

  In Section~\ref{sec.cubic}, you'll find at most
  three roots of a 3rd degree polynomial (called a cubic).  Complete
  the code in \code{cubic.py}.

\item \textbf{Roots of quartic and quintic:}

  If there's time remaining in the
  atelier, Sections~\ref{sec.quartic} and~\ref{sec.quintic} deal with
  4th and 5th order polynomials.  Even if there's not enough time to
  finish all the exercises in this atelier, you may continue the work
  on your own time, because your account of GitHub will remain in
  place as long as you do not delete it.
  
\end{enumerate}


Good Luck! And Happy Coding.


% LocalWords:  atelier Atelier GitHub quartic quintic
