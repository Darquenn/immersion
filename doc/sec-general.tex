\section{Review of Polynomials (Optional)}
\label{sec.general}

Your goal in the next several sections is to find \emph{root} of \emph{polynomials}
But first, what is a polynomial? And then, what is a root?

\begin{definition}{polynomial of degree n}{}
  A \emph{polynomial} of \emph{degree n} is a function of a single variable, usually $x$,
  of the form.
  \[P(x) = \alpha_n x^n + \alpha_{n-1} x^{n-1} + \cdots + \alpha_2 x^2 + \alpha_1 x + \alpha_0 \]

  where $\alpha_n,  \alpha_{n-1} x^{n-1}, \ldots, \alpha_0$ are called coefficients and may be integers, rational numbers, or real numbers.
\end{definition}

An example of a 4th degree polynomial is
\[P(x) = 2 x^4 + 5 x^3 + x^2 - x + 1\,.\]


Any of the coefficients after the leading one may be 0, in which case the term is generally omitted.
An example of a 3th degree polynomial where the coefficients of the $x^2$ term is 0 is as follows.
\[P(x) = 5 x^3 + x - 1\,.\]

\begin{definition}{root}{}
  If $P(x)$ is a polynomial, and $r$ is a number for which $P(r)=0$, then $r$ is called a \emph{root}
  of the polynomial.
\end{definition}

The polynomial $P(x) = x^3 - x$ as three roots, 1, -1, and 0.  We know this because
\begin{align*}
  P(1) &= (1)^3 - 1 
  = 1 - 1 
  = 0\\[3pt]
  P(-1) &= (-1)^3 - (-1) 
  = -1 + 1 
  = 0\\[3pt]
  P(0) &= (0)^3 - 0
  = 0 - 0 
  = 0  
\end{align*}

\begin{theorem}{Fundamental Theorem of Algebra}{ftoa}
  Every non-constant polynomial of degree~1 or higher has a root.
\end{theorem}

The Fundamental Theorem of Algebra claims that every polynomial.  However, sometimes the root is complex.
For example the polynomial, $P(x) = x^2 + 1$, has two comples roots.
In this atelier, we we only attempt to determine the real roots, ignoring the complex roots.
For this reason, we will not be able to find all the roots of some polynomials.

\begin{corollary}{Roots with Multiplicity}{n.roots}
  A polynomial of degree $n\ge 1$ has $n$ roots, some of which may be equal.
\end{corollary}
\begin{proof}

  [By Induction]
  
  \begin{itemize}
    
  \item \textbf{Base case:} If $n=1$, then Theorem~\ref{th.ftoa}
    guarantees it has a root; call it $r_1$.  Thus it is of the form
    $P(x)=\alpha (x-r_1)$.
  \item \textbf{Inductive case:} Suppose $n>1$.  Suppose every polynomial, $Q(x)$ of degree
    $n$ has $n$-many roots, $r_1,\ldots,r_n$, and can thus be written as:
    \[Q(x) = \alpha_1 (x - r_1) (x - r_2) \cdots (x - r_n)\]
    Now consider a polynomial, $P(x)$, of
    degree~$n+1$.  Theorem~\ref{th.ftoa} guarantees $P(x)$ has a root;
    call it $r_{n+1}$, thus $(x-r_{n+1})$ is a factor of $P(x)$ so there
    exists a polynomial, $Q(x)=$, of degree~$n$ such that
    $P(x) =  Q(x) (x-r_{n_1})$.
    But by inductive hypothesis, $Q(x)$ can be written
    as \[Q(x) = \alpha_1 (x - r_1) (x - r_2) \cdots (x - r_n)\]
    So, \[P(x) = \alpha_1 (x - r_1) (x - r_2) \cdots (x - r_n) (x-r_{n+1})\]

    Thus $P(x)$ has $n+1$ roots.
  \end{itemize}
\end{proof}

Corollary~\ref{cor.n.roots} can be seen as a model of how we will find
roots in this attempt.  Given a degree~$n$ polynomial (for $2<n\leq
5$), if we successfully find a root, $r_n$, then we will factor
$(x-r_n)$ out of the polynomial, giving us new polynomial but with
degree~$n-1$, which we can solve by the same approach.  This process
continues until we reach degree~$n=2$ which we solve using the
quadratic formula.
